\section{Conclusions and Discussion}

% \textbf{Summary} do not need to specify
This paper presents a psychophysical experiment used for evaluating the effectiveness of DoF blur on visual discomfort on immersive HMDs like the Oculus. We find that DoF blur decreases overall visual discomfort on HMDs, when tested on dynamic, interactive scenes.\\

% \textbf{Analysis of effectiveness of DOF blur in HMD}: do not need to specify
We implemented dynamically adjusted DoF of the rendered scene on the Oculus such that the centre of the virtual view-port formed by combining the stereo 3D screens always remains in focus to the user. Our DoF implementation discourages the users focusing in regions outside the centre of the screen as well as discouraging focusing on objects that are at significantly different depths to the current focal distance, as they are blurred. This mitigates the accommodation-vergence conflict in the human visual system and mimics peripheral blur that is absent on typical LCD viewing.\\

In general stereo displays (e.g. LCD), limiting eye movement to only focusing on the centre of a screen will limit users' spatial degrees of freedom. However, in HMDs, the majority of spatial movement will occur through head movements (i.e., changing the orientation of the Oculus to look somewhere else) rather than through eye movements, compensating for this drawback.\\

There were a total of two participants who stated \say{(they) felt equally sick on both sessions}, and had consistent responses to their discomfort symptoms on each session. This indicates that there are people for whom DoF blur is not always relevant for reducing visual discomfort on HMDs: reducing the contribution of the accommodation-vergence conflict does not significantly affect overall their visual discomfort. \\


We used a common HMD, with display settings optimised for an average human, with a mouse and keyboard control scheme implemented in many VR applications. As this control scheme is not ubiquitous and the average settings will not work optimally for all users, further experiments should allow more user fine-tuning of the display and control settings.\\

% It is not required to mentioned
% Further experiments for determining discomfort should be conducted on climate controlled rooms to eliminate potential external variables. \\

